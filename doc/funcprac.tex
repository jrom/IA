%!TEX root = document.tex

El codi principal de la pràctica es troba en la classe \emph{SquareBoard.java}. Per tal d'efectuar els experiments, hem creat un programa principal que s'executa amb la comanda \emph{make run} i prova els diferents escenaris.

\begin{lstlisting}[caption=Comandes, label=comandes, language=Bash]
make # Compila els .java
make run # Executa el programa principal
make clean # Esborra fitxers font
\end{lstlisting}

El menú es mostra de la següent manera:

\begin{verbatim}
--------------------- Selecció d'experiments -------------
Experiment 1: Comparació solucions inicials
Experiment 2: Estudi creixement de K
Experiment 3: Efectes de la restricció addicional
Experiment 6: Ponderació entre heurístics
Experiment 7: Parametre It. Totals de SimulatedAnnealing
Experiment 8: Parametre It. per pas de SimulatedAnnealing
Experiment 9: Parametre K de SimulatedAnnealing
Experiment 10: Parametre Lambda de SimulatedAnnealing
Experiment 11: Comparació heurístics h1 i h2
----------------------------------------------------------
Introdueix el número d'experiment desitjat (de 1 a 11)
\end{verbatim}

Tant sols cal introduir el número d'experiment i prémer Intro per executar el codi de proves. En alguns casos es genera un taulell sobre el qual es realitzen 10 simulacions i s'extreu la mitjana, en altres casos es genera més d'un taulell. Hem escollit cada escenari de proves segons hem cregut que reflexa més fidelment els valors mitjans.