En aquesta pràctica es planteja optimitzar l'organització de les rutes d'autobús del poble de \emph{Squaretown}. Es tracta d'una ciutat urbanitzada de forma reticular quadrada (quadrícula de 20 x 20). Per facilitar l'anàlisis del problema, i sense pèrdua de generalitat, d'ara en endavant parlarem d'un taulell quadrat, amb files, columnes i caselles. Es dóna la ubicació fixa de les dues úniques cotxeres d'autobusos que hi haurà disponibles, d'on unívocament començarà o acabarà qualsevol ruta. Aquestes es troben a les caselles (1,1) i (19,19). La nostra primera decisió consisteix en determinar que la generació de rutes es farà sempre des de la cotxera (1,1) a la (19,19). Òbviament, per qualsevol escenari real això seria fàcilment canviable paramètricament.

El problema plantejat és definir K $\epsilon$ \{2-10\} rutes d'autobús que connecten P $\epsilon$ \{10-50\} parades obtingudes aleatòriament. Tenim dos criteris a optimitzar. El primer és la minimització del recorregut total de les rutes. El segon és la maximització de la similitud de les distàncies entre parades.

Una restricció addicional (i opcional) assenyala que totes les rutes tinguin almenys $\frac{P}{2*K}$ parades.

Els algoritmes de cerca local que usarem per realitzar aquests càlculs són el \textbf{Hill Climbing} i el \textbf{Simulated Annealing}. Amb ells haurem d'experimentar modificant paràmetres, usant diferents configuracions inicials, aplicant o traient restriccions i provant els diferents heurístics per veure quins són els que ofereixen millors resultats.


% 
% \section{Intro}
% 
% \subsection{Blablabla}
% asdfasdf
% 
% \subsubsection{Lol}
% 
% Amb textmate es compila femt pometa-R
% 
% Ah, i has de deixar una línia en blanc entre dos paràgrafs si vols que surtin separats, sinó anirà a continuació...
% 
% Negreta: \textbf{text en negreta}
% 
% Cursiva: \textbf{Text en cursiva}

% Variable o similar: \codi{x.y}
% 
% \begin{lstlisting}[label=compi, caption=Comanda de compilació, language=Java]
% codi
% llarg
% \end{lstlisting}
% 
