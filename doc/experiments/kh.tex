%!TEX root = ../document.tex

Com que volem una funció heurística que combini els dos paràmetres de qualitat esmentats (apartat ~\ref{sub:heuristic3}
), cal saber com conbinar-los. 

Per tal de calcular una constant de ponderació $k_{h_{21}}$ hem realitzat un experiment que consisteix en observar el valor dels heurístics 1 (recorregut total de les rutes) i 2 (distància entre rutes homogènia) depenent de com els ponderéssim, i evidentment fent que l'algorisme intenti minimitzar la combinació d'ambdós.

Hem utilitzat els paràmetres que han donat millor solucions fins al moment per dur a terme aquest experiment, o sigui:
\begin{itemize}
	\item La solució inicial de qualitat
	\item L'algorisme de Simulated Annealing amb els paràmetres trobats a l'apartat ~\ref{sec:expSA}
	\item Tots els valors de ponderació entre $0.1$ i $1.0$
\end{itemize}

La fórmula per ponderar els heurístics $h_1$ i $h_2$ és la següent:

\begin{center}
	\large
	\[
		h_3 = (1 - k_{h_{21}}) \times h_1 + k_{h_{21}} \times h_2
	\]
\end{center}

\imatge{images/gkh.png}{Valor dels heurístics 1 i 2 amb diferents ponderacions}

Es pot observar a la figura ~\ref{fig:{images/gkh.png}} que com és d'esperar l'heurístic 1 comença oferint un resultat pobre (un valor alt) ja que no se l'està tenint gens en compte, i acaba per estabilitzar-se amb valors de $k_{h_{21}} \in \{ 0.5 .. 0.9\}$. Per altra banda l'heurístic 2 mostra millor resultats amb els valors mitjos, on els dos heurístics es combinen paritàriament. Per això s'ha fixat per a la resta de proves que el valor ideal de $k_{h_{21}}$ sigui \textbf{0.5}.
