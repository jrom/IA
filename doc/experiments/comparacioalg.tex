%!TEX root = ../document.tex

En aquest experiment hem provat en 10 taulers diferents el resultat de fer 10 cerques amb l'algorisme Hill Climbing i 10 amb el de \emph{Simulated Annealing}. El resultat que s'observa és el promig de totes les execucions en tots els taulers. S'ha escollit usar els paràmetres òptims de Simulated Annealing (\ref{sec:expSA}), la solució de qualitat i l'heurístic combinat.

Es pot observar en la figura \ref{fig:{images/hcvssa.png}} que en els tres heurístics donen un millor resultat les cerques amb \emph{Simulated Annealing}, i la diferència entres els heurístics és bastant equivalent, el que mostra gràficament que la ponderació entre $h_1$ i $h_2$ està ben calibrada.

\imatge{images/hcvssa.png}{Hill Climbing contra Simulated Annealing}

A més de veure quin és l'algorisme que proporciona més qualitat a la solució, també observem que \emph{Hill Climbing} és moltm és resolutiu, ja que assoleix un resultat prou similar amb una fracció molt petita de passos, i sobretot és més eficient perquè la diferència de nodes expandits i \emph{elapsed time} entre els algorismes és aclaparadorament millor en l'algorisme de \emph{Hill Climbing}.
