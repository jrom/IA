%!TEX root = document.tex

Per tal de justificar decisions preses durant el transcurs de la pràctica, o bé per comprovar-ne l'eficàcia, s'han anat duent a terme experiments que a continuació llistem i mostrem en profunditat.

El primer que es demanava era calibrar l'algorisme de \emph{Simulated Annealing}, i per tant provar possibles valors pels 4 paràmetres. Es pot consultar a l'apartat ~\ref{sec:expSA}.

A continuació a la secció ~\ref{sec:expsolini}
 es comparen les diferents solucions inicials, observant-ne la qualitat i el temps d'execució necessari.

També s'analitza la influència del paràmetre $K$ (nombre de rutes) en el cost i la qualitat de la cerca a l'apartat ~\ref{sec:expk}.

En quant als paràmetres de la cerca, comencem comparant els efectes d'usar l'heurístic de la distància total contra el de les distàncies entre parades semblants (\ref{sec:comph1h2}). Després calculem el valor òptim de la ponderació d'aquests dos en un tercer heurístic (\ref{sec:expkh}). Finalment comparem els dos algorismes de cerca: Hill Climbing i Simulated Annealing (amb els paràmetres òptims experimentats) a l'apartat \ref{sec:comparacioalg}.

Com a afegit, provem quin és el resultat si es prenen consideració la restricció addicional (\ref{sec:restadd}).


\newpage
\section{Paràmetres de Simulated Annealing} % (fold)
\label{sec:expSA}
%!TEX root = ../document.tex

L'algorisme \emph{Simulated Annealing} pren quatre paràmetres per definir el seu comportament. Aquests serveixen per indicar de quina manera ha d'efectuar l'\emph{escalfament} que orientarà l'algorisme.

Els paràmetres són:

\begin{itemize}
	\item Iteracions
	\item Iteracions per cada pas de temperatura
	\item Paràmetre K
	\item Paràmetre $\lambda$
\end{itemize}

A continuació mostrem els resultats d'experimentar amb possibles valors de cadascun d'ells. Per aquests experiments s'ha triat fixar tres dels valors amb els valors per defecte que assigna AIMA si no s'especifiquen (10000, 100, 20 i 0.045 respectivament) i en el valor experimentat provar un ampli ventall de possibilitats, per observar gràficament quina és la opció que ofereix millor resultats.

En els gràfics que es mostren a continuació tots els valors s'han normalitzat per tal de poder observar si l'efecte és positiu o negatiu en cada un dels paràmetres observats. Simplement convé observar quins valors creixen i quins decreixen amb diferents possibilitats en els paràmetres de \emph{Simulated Annealing}.

Per a executar aquestes proves hem escollit la solució inicial de qualitat. S'ha executat 10 vegades la cerca per cada tauler i valor del paràmetre en concret, amb 10 taulers diferents. Els gràfics estan fets amb el promig d'aquests 100 resultats (per valor per paràmetre).

\newpage
\subsection{Iteracions màximes} % (fold)
\label{sub:iteracions}

Amb aquest paràmetre definim el número màxim d'iteracions, de certa manera limitant tota la resta de factors obtinguts per l'algorisme perquè no desemboqui en massa càrrega de treball.

\imatge{images/parsa_steps.png}{Influència del paràmetre iteracions màximes}

A la figura ~\ref{fig:{images/parsa_steps.png}} observem que a mesura que el número màxim d'iteracions creix (en un rang entre 500 i 10000) creix també linealment el número de nodes expandits i l'\emph{elapsed time}. D'una manera molt més erràtica i menys clara es pot veure que els valors dels tres heurístics es mouen d'una forma similar, i el número de passos fins a la millor solució trobada també segueix una mica el mateix patró. Sense que sigui una diferència molt gran, al punt de les \textbf{5500} iteracions s'observa una davallada dels heurístics i el número de passos, i en la zona fins a les 7500 iteracions és on s'observen els millors resultats globals.

Per aquest fet s'ha escollit que el valor òptim d'aquest paràmetre és \textbf{5500} iteracions.

% subsection iteracions (end)

\newpage
\subsection{Iteracions per cada pas de temperatura} % (fold)
\label{sub:iteracions_per_cada_pas}

L'algorisme de \emph{Simulated Annealing} va variant el valor de la temperatura que empra per guiar-se, i en cada temperatura diferent realitza un número d'iteracions. Aquest és el paràmetre que estudiem ara. En el codi per defecte d'AIMA s'utilitza un valor de 100 iteracions, i per aquest motiu hem decidit provar els valors entre 20 i 200 amb increments de 10 per escollir-ne l'òptim.

\imatge{images/parsa_limit.png}{Influència del paràmetre iteracions per pas}

En aquest cas no tenim cap increment lineal sino un valor constant, que és el número de nodes expandits. Aquest factor no és afectat pel número d'iteracions per pas. Per altra banda, observem que el número de passos sí que disminueix a mesura que el paràmetre augmenta, i l'\emph{elapsed time} també tendeix a decréixer. Els heurístics tenen un comportament menys clar, tot i que en aquesta gràfica s'observa que al punt de les \textbf{150} iteracions tots els valors coincideixen en un mínim que fa clara l'elecció a prendre.

Tot i que el valor dels heurístics assoleix un mínim absolut a la zona dels 60-80, la combinació de tots els factors és clarament guanyadora en el punt dels 150. Si l'objectiu fos conseguir solucions amb un número de passos petit, es podria escollir el número més gran que minimitza aquest valor. En el nostre cas ens és irrellevant el camí per arribar a la millor solució.

% subsection iteracions_per_cada_pas (end)

\newpage
\subsection{Paràmetre K} % (fold)
\label{sub:parametre_k}

El valor de $K$ determina quant triga la temperatura en començar a descendir. El valor per defecte d'AIMA és de 20 i nosaltres hem provat totes les possibilitats entre 10 i 200 de 10 en 10.

\imatge{images/parsa_K.png}{Influència del paràmetre K}

Podem comprovar que a els heurístics mostren una tendència a decréixer a mesura que augmenta K, i en canvi el número de passos incrementa. El valor de l'\emph{elapsed time} es comporta d'una manera irregular, tot i que amb els valors dels extrems aquest augmenta molt.

Si el que es vol és millorar la qualitat dels heurístics, el punt òptim es troba quan $k = 110$ ja que aquests assoleixen un mínim, mentre que el número de passos tampoc és molt gran i l'\emph{elapsed time} també és moderat. Altres valors positius són el 120, 140 i 170, amb uns valors similars.

Escollim com a valor per defecte del paràmetre $K$ \textbf{110}.


% subsection parametre_k (end)
\newpage
\subsection{Paràmetre $\lambda$} % (fold)
\label{sub:parametre_lambda}

El paràmetre $\lambda$ és el que marca la pendent del refredament de la funció, o sigui com n'és de ràpid aquest refredament un cop comença. A partir dels gràfics observats a les transparències sobre AIMA i el valor per defecte que aquesta classe defineix per $\lambda$, decidim provar els possibles valors entre $0.0005$ i $0.128$ incrementant a base de multiplicar per dos. D'aquesta manera podem contemplar el comportament de la cerca amb un ampli ventall de valors però sense haver de provar-ne milers. Els resultats són bastant clars en el següent gràfic:

\imatge{images/parsa_lambda.png}{Influència del paràmetre $\lambda$}

Degut a la naturalesa exponencial dels valors de l'eix X s'han d'interpretar els resultats exponencialment. S'observa que l'\emph{elapsed time} i el número de passos fins a la solució decrementa molt amb els primers valors de $\lambda$, i que tots els heurístics prenen millors valors amb les $\lambda$ més petites. Per trobar u ncompromís es pot escollir una $\lambda$ de \textbf{0.002} que obté molt bons valors en els heurístics i l'\emph{elapsed time} ha caigut bastant respecte els valors més petits.

\subsection{Resum dels paràmetres de Simulated Annealing} % (fold)
\label{sub:resum_parametres}

Després de comentar un per un els paràmetres requerits per l'algorisme de Simulated Annealing, mostrem les conclusions d'aquests experiments: Hem partit de la informació bàsica proporcionada pels apunts d'AIMA i el seu propi codi, on hem observat quins valors són raonables aplicar a aquest algorisme. A partir d'aquí hem decidit provar-ne tants com fos oportú per tenir la certesa de trobar l'òptim. Normalment hem provat 20 valors diferents per cada paràmetre, i ens hem quedat amb el que assolia millors resultats tant en el valor dels heurístics com (en menys mesura) els valors de l'\emph{elapsed time} i el número de passos.
\begin{figure}[ht]
  \caption{Taula amb els valors dels paràmetres per Simulated Annealing}
	\label{fig:resumSA}
  \begin{center}
    \begin{tabular}{| l | c | c | c |}
      \hline
      						& \textbf{Valor AIMA}    & \textbf{Valors} & \textbf{Resultat} \\ \hline
      Iteracions màximes 	& 10000 & 500 .. 10000 		& 5500 \\ \hline
      Iteracions per pas 	& 100 	& 20 .. 200 		& 150 \\ \hline
      K 					& 20 	& 10 .. 200 		& 110 \\ \hline
      $\lambda$ 			& 0.045 & 0.0005 .. 0.128 	& 0.02 \\ \hline
    \end{tabular}
  \end{center}
\end{figure}
% subsection resum_parametres (end)

Donada la precisió amb la que hem intentat fer els experiments (amb 10 iteracions per cada possible valor) i repetint tot l'experiment en 10 taulers diferents (per tant 100 execucions per cada valor de cada paràmetre), amb un total de 8000 cerques per elaborar aquests quatre gràfics (evidentment automatitzades amb el codi de la classe \texttt{Main}), creiem que la diferència entre els resultats obtinguts i els paràmetres per defecte d'AIMA (notable tot i que dins dels paràmetres raonables) han de ser deguts a la peculiaritat del nostre problema. Tots els valors que hem trobat han estat dins del que cabia esperar i han millorat la qualitat de la cerca.

\newpage
% subsection parametre_lambda (end)
% section expSA (end)

\section{Influència de les solucions inicials} % (fold)
\label{sec:expsolini}
%!TEX root = ../document.tex

Tal com indica la teoria de la cerca local, obtenir una bona solució inicial fa més fàcil l'obtenció de millors resultats. Ho hem comprovat realitzant 10 execucions amb una mateixa configuració de parades. En la següent taula es pot veure el promig d'aquestes 10 execucions. En aquest cas, només s'intenta maximitzar l'heurístic 1.

Recordem que la solució inicial 1 és aquella que incorpora coneixement del problema i intenta optimitzar d'entrada la configuració inicial. La solució 2 és aquella generada immediatament a partir de la configuració aleatòria de parades.

Valor de l'heurístic 1 de la solució inicial 1: \textbf{288.0}

Valor de l'heurístic 1 de la solució inicial 2: \textbf{384.0}

\begin{verbatim}
	+-------------------------+---------+----------+
	| Experiment              | Elapsed |    H1    |
	+-------------------------+---------+----------+
	| HC - Solució Inicial 1  |       8 |   256,00 |
	| SA - Solució Inicial 1  |    1209 |   229,00 |
	| HC - Solució Inicial 2  |       2 |   308,00 |
	| SA - Solució Inicial 2  |    1215 |   237,20 |
	+-------------------------+---------+----------+
	
	HC Hill Climbing
	SA Simulated Annealing
\end{verbatim}
		
L'ús de la solució inicial 1 millora els resultats de la solució 2 en un \textbf{17\%} en el cas de l'algorisme Hill Climbing i en un \textbf{4\%} en el cas de Simulated Annealing. D'això s'infereix que una bona solució inicial porta millors resultats pel Hill Climbing que pel Simulated Annealing.

Per últim, hem volgut comprovar com es podia millorar els resultats del Hill Climbing mitjançant una estratègia de reinici aleatori (tal com ha estat explicat a classe). Per això hem executat l'algorisme de Hill Climbing 50 vegades, cadascuna amb una solució inicial aleatòria diferent (solució inicial 3) per acabar quedant-nos amb el mínim H1 obtingut. Aquest experiment l'hem realitzat 10 vegades i n'hem fet el promig en la següent taula:

Valor de l'heurístic 1 de la solució inicial 1: \textbf{248.0}

Valor de l'heurístic 1 de la solució inicial 2: \textbf{360.0}

\begin{verbatim}
	+-------------------------------+----------+----------+
	| Experiment                    | H1 en HC | H1 en SA |
	+-------------------------------+----------+----------+
	| Solució Inicial 1             |   123,00 |   111,00 |
	| Solució Inicial 2             |   134,00 |   113,00 |
	| Solució Inicial 3             |   145,00 |   115,00 |
	| RRHC - Sol. Inicial 3 x 50    |   119,00 |        - |
	+-------------------------------+----------+----------+

	HC Hill Climbing
	SA Simulated Annealing
	RRHC Random Restarting Hill Climbing
\end{verbatim}

                                          
Es pot comprovar que la solució de qualitat aquí segueix millorant les aleatòries però quan usem l'estratègia de reinici aleatori (amb 50 execucions) obtenim per un cost, fins i tot inferior al del Simulated Annealing, resultats semblants a aquest amb la solució de qualitat. En particular, obtenim un resultat 4 punts millor que la solució obtinguda amb Hill Climbing + la solució inicial de qualitat i tant sols 8 punts pitjor que Simulated Annealing + solució inicial de qualitat, o bé 4 punts pitjor que Simulated Annealing amb solució inicial aleatòria.


	
% section expsolini (end)
\newpage
\section{Com varia la dificultat del problema en funció de K?} % (fold)
\label{sec:expk}
%!TEX root = ../document.tex


Per comprovar com afecta el número de rutes en la dificultat del problema hem aplicat 10 vegades els algorismes de Hill Climbing i Simulated Annealing sobre una configuració inicial aleatòria de 30 parades i n'hem obtingut la mitjana. Els resultats es mostren als següents gràfics (que han estat normalitzats).

\imatge{images/HC_K.png}{Heurístic 1, temps d'execució (\emph{elapsed}) i passos (\emph{steps}) de l'algorisme Hill Climbing en funció de K.} 

En el gràfic de la Figura ~\ref{fig:{images/HC_K.png}} es pot comprovar com evolucionen els paràmetres en l'algorisme de Hill Climbing a mesura que va augmentant el nombre de rutes (K). La primera observació és el pic del temps d'execució de l'algorisme quan la K és 2 (es troba en l'ordre de 30 u.t.) i es redueix progressivament a partir de 3 fins mantenir-se estable per la resta de valors de K. Els passos de l'algorisme canvien relativament poc (es mouen en l'interval de 10 a 15) i el seu màxim es troba en la K = 5, que realitza 15 passos. El recorregut de les rutes (Heurístic 1) augmenta progressivament a mesura que la K augmenta, del qual se'n dedueix que li resulta més fàcil a l'algoritme trobar valors millors quan es tenen menys rutes.


\imatge{images/SA_K.png}{Heurístic 1, temps d'execució (\emph{elapsed}) i passos (\emph{steps}) de l'algorisme Simulated Annealing en funció de K.}  

En el cas del Simulated Annealing (Figura ~\ref{fig:{images/SA_K.png}}), s'observa altre cop un pic en el temps d'execució de l'algorisme per K = 2 (en particular 4344 u.t.). El temps d'execució baixa progressivament fins que veu el seu mínim en K = 5 (3094 u.t.) i torna a pujar fins quasi 5000 u.t. per K = 10. Els passos segueixen més o menys el mateix patró de creixement que el temps d'execució. El recorregut de les rutes, però, creix igual que el Hill Climbing progressivament a mesura que s'afegeixen més rutes al problema. Tanmateix, per qualsevol K, sempre millora el valor de l'algorisme de Hill Climbing per un valor entre 75 i 125.

Així doncs, s'extreu com a conclusió que la K augmenta la dificultat del problema en ambdós casos i empitjora els resultats en els dos algorismes, tot i que el temps d'execució augmenta de forma més evident en el Simulated Annealing.

% section expk (end)


% section comparacioalg (end)
\newpage
\section{Comparació dels heurístics} % (fold)
\label{sec:comph1h2}
%!TEX root = ../document.tex

Per tal de poder analitzar el comportament de l'algorisme amb els dos heurístics (sense combinar), comprovem quins resultats obtenim fent-ne servir cadascun d'ells.

En el següent experiment hem fet servir 10 taulers diferents, i per cada tauler hem fet 10 iteracions per cada heurístic, combinant les dades fins tenir-ne dues series.

La solució inicial escollida ha estat la de qualitat i l'algorisme Simulated Annealing amb els paràmetres comentats anteriorment.


\imatge{images/h1vsh2.png}{Resultats amb h1 i h2}

A la Figura~\ref{fig:{images/h1vsh2.png}} s'observa que com és d'esperar les execucions amb la cerca guiada per l'heurístic 1 obtenen un valor d'aquest més baix que les que han estat guiades per l'heurístic 2, tot i que la diferència no és molt gran. En aquest sentit, escollir un heurístic no perjudica el valor de l'altre.

Per altra banda, observem un efecte molt contundent en aquest gràfic, i és que l'heurístic 1 consegueix resultats amb molts més passos per arribar a la solució però la meitat del temps d'execució. Finalment, comentar que l'heurístic 2 és el que consegueix (per poca diferència) el millor valor per l'heurístic 3, que no s'ha emprat per guiar la cerca sino només per il·lustrar.

% section comph1h2 (end)


% Experiment de ponderacio de KH21
\newpage % Per ara ho faig així perquè quedi tot l'experiment en una pàgina i no se'n vagi la imatge porai
\section{Paràmetre de ponderació dels heurístics} % (fold)
\label{sec:expkh}
%!TEX root = ../document.tex

\section{Cerca del paràmetre de ponderació dels heurístics} % (fold)
\label{sec:expkh}

Lorem ipsum dolor sit amet, consectetur adipisicing elit, sed do eiusmod tempor incididunt ut labore et dolore magna aliqua. Ut enim ad minim veniam, quis nostrud exercitation ullamco laboris nisi ut aliquip ex ea commodo consequat. Duis aute irure dolor in reprehenderit in voluptate velit esse cillum dolore eu fugiat nulla pariatur. Excepteur sint occaecat cupidatat non proident, sunt in culpa qui officia deserunt mollit anim id est laborum.
% section expkh (end)
% section expkh (end)


\newpage
\section{Comparació d'algorismes} % (fold)
\label{sec:comparacioalg}
%!TEX root = ../document.tex

En aquest experiment hem provat en 10 taulers diferents el resultat de fer 10 cerques amb l'algorisme Hill Climbing i 10 amb el de \emph{Simulated Annealing}. El resultat que s'observa és el promig de totes les execucions en tots els taulers. S'ha escollit usar els paràmetres òptims de Simulated Annealing (\ref{sec:expSA}), la solució de qualitat i l'heurístic combinat.

Es pot observar en la figura \ref{fig:{images/hcvssa.png}} que en els tres heurístics donen un millor resultat les cerques amb \emph{Simulated Annealing}, i la diferència entres els heurístics és bastant equivalent, el que mostra gràficament que la ponderació entre $h_1$ i $h_2$ està ben calibrada.

\imatge{images/hcvssa.png}{Hill Climbing contra Simulated Annealing}

A més de veure quin és l'algorisme que proporciona més qualitat a la solució, també observem que \emph{Hill Climbing} és moltm és resolutiu, ja que assoleix un resultat prou similar amb una fracció molt petita de passos, i sobretot és més eficient perquè la diferència de nodes expandits i \emph{elapsed time} entre els algorismes és aclaparadorament millor en l'algorisme de \emph{Hill Climbing}.



\section{Com varia el problema aplicant la restricció addicional?} % (fold)
\label{sec:restadd}
%!TEX root = ../document.tex

Se'ns demana estudiar el comportament dels algorismes quan s'aplica la restricció addicional descrita a l'enunciat. Aquesta consisteix en limitar l'espai de cerca per tal que totes les rutes tinguin almenys $\frac{P}{2*K}$ parades.

L'experiment es basa en l'execució sobre un taulell generat aleatòriament per la solució inicial aleatòria i on s'apliquen els algorismes de Hill Climbing i Simulated Annealing amb i sense restriccions. Els algorismes s'executen 10 vegades sobre el taulell per extreure'n la mitjana dels resultats obtinguts. En la següent taula es pot veure la informació del temps d'execució, els nodes expandits, els passos i el recorregut total de les rutes (heurístic 1). L'algorisme optimitza l'heurístic 1.

\begin{verbatim}
+------------+---------+-------+-------+--------+
| Experiment | Elapsed | Nodes | Steps |   H1   |
+------------+---------+-------+-------+--------+
| HC Rest    |      19 |     4 |     3 | 348,00 |
| SA Rest    |    3626 |  5501 |   233 | 305,20 |
| HC No Rest |       1 |     2 |     1 | 298,00 |
| SA No Rest |    3209 |  5501 |   150 | 293,00 |
+------------+---------+-------+-------+--------+
                                               
HC Rest: Hill Climbing amb o sense rest. addicional.
SA Rest: Simulated Annealing amb o sense rest. addicional.
\end{verbatim}  

Com es pot comprovar, lluny de facilitar el problema el complica. Tant l'algorisme de Hill Climbing com el de Simulated Annealing empitjoren els seus resultats. Mentre el temps d'execució, els passos i els nodes empitjoren en el Hill Climbing, en el cas de Simulated Annealing els nodes expandits es mantenen invariables, tot i que els passos i el temps d'execució també creixen quan s'aplica la restricció. Deduïm doncs que en aquest cas resulta més convenient permetre que es generin sempre tots els successors sense la restricció en l'espai de cerca.                                    

% section restadd (end)
