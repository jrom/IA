%!TEX root = document.tex

Els experiments a fer són els següents:

\section{Paràmetres de Simulated Annealing} % (fold)
\label{sec:expSA}
%!TEX root = ../document.tex

L'algorisme \emph{Simulated Annealing} pren quatre paràmetres per definir el seu comportament. Aquests serveixen per indicar de quina manera ha d'efectuar l'\emph{escalfament} que orientarà l'algorisme.

Els paràmetres són:

\begin{itemize}
	\item Iteracions
	\item Iteracions per cada pas de temperatura
	\item Paràmetre K
	\item Paràmetre $\lambda$
\end{itemize}

A continuació mostrem els resultats d'experimentar amb possibles valors de cadascun d'ells. Per aquests experiments s'ha triat fixar tres dels valors amb els valors per defecte que assigna AIMA si no s'especifiquen (10000, 100, 20 i 0.045 respectivament) i en el valor experimentat provar un ampli ventall de possibilitats, per observar gràficament quina és la opció que ofereix millor resultats.

En els gràfics que es mostren a continuació tots els valors s'han normalitzat per tal de poder observar si l'efecte és positiu o negatiu en cada un dels paràmetres observats. Simplement convé observar quins valors creixen i quins decreixen amb diferents possibilitats en els paràmetres de \emph{Simulated Annealing}.

Per a executar aquestes proves hem escollit la solució inicial de qualitat. S'ha executat 10 vegades la cerca per cada tauler i valor del paràmetre en concret, amb 10 taulers diferents. Els gràfics estan fets amb el promig d'aquests 100 resultats (per valor per paràmetre).

\newpage
\subsection{Iteracions màximes} % (fold)
\label{sub:iteracions}

Amb aquest paràmetre definim el número màxim d'iteracions, de certa manera limitant tota la resta de factors obtinguts per l'algorisme perquè no desemboqui en massa càrrega de treball.

\imatge{images/parsa_steps.png}{Influència del paràmetre iteracions màximes}

A la figura ~\ref{fig:{images/parsa_steps.png}} observem que a mesura que el número màxim d'iteracions creix (en un rang entre 500 i 10000) creix també linealment el número de nodes expandits i l'\emph{elapsed time}. D'una manera molt més erràtica i menys clara es pot veure que els valors dels tres heurístics es mouen d'una forma similar, i el número de passos fins a la millor solució trobada també segueix una mica el mateix patró. Sense que sigui una diferència molt gran, al punt de les \textbf{5500} iteracions s'observa una davallada dels heurístics i el número de passos, i en la zona fins a les 7500 iteracions és on s'observen els millors resultats globals.

Per aquest fet s'ha escollit que el valor òptim d'aquest paràmetre és \textbf{5500} iteracions.

% subsection iteracions (end)

\newpage
\subsection{Iteracions per cada pas de temperatura} % (fold)
\label{sub:iteracions_per_cada_pas}

L'algorisme de \emph{Simulated Annealing} va variant el valor de la temperatura que empra per guiar-se, i en cada temperatura diferent realitza un número d'iteracions. Aquest és el paràmetre que estudiem ara. En el codi per defecte d'AIMA s'utilitza un valor de 100 iteracions, i per aquest motiu hem decidit provar els valors entre 20 i 200 amb increments de 10 per escollir-ne l'òptim.

\imatge{images/parsa_limit.png}{Influència del paràmetre iteracions per pas}

En aquest cas no tenim cap increment lineal sino un valor constant, que és el número de nodes expandits. Aquest factor no és afectat pel número d'iteracions per pas. Per altra banda, observem que el número de passos sí que disminueix a mesura que el paràmetre augmenta, i l'\emph{elapsed time} també tendeix a decréixer. Els heurístics tenen un comportament menys clar, tot i que en aquesta gràfica s'observa que al punt de les \textbf{150} iteracions tots els valors coincideixen en un mínim que fa clara l'elecció a prendre.

Tot i que el valor dels heurístics assoleix un mínim absolut a la zona dels 60-80, la combinació de tots els factors és clarament guanyadora en el punt dels 150. Si l'objectiu fos conseguir solucions amb un número de passos petit, es podria escollir el número més gran que minimitza aquest valor. En el nostre cas ens és irrellevant el camí per arribar a la millor solució.

% subsection iteracions_per_cada_pas (end)

\newpage
\subsection{Paràmetre K} % (fold)
\label{sub:parametre_k}

El valor de $K$ determina quant triga la temperatura en començar a descendir. El valor per defecte d'AIMA és de 20 i nosaltres hem provat totes les possibilitats entre 10 i 200 de 10 en 10.

\imatge{images/parsa_K.png}{Influència del paràmetre K}

Podem comprovar que a els heurístics mostren una tendència a decréixer a mesura que augmenta K, i en canvi el número de passos incrementa. El valor de l'\emph{elapsed time} es comporta d'una manera irregular, tot i que amb els valors dels extrems aquest augmenta molt.

Si el que es vol és millorar la qualitat dels heurístics, el punt òptim es troba quan $k = 110$ ja que aquests assoleixen un mínim, mentre que el número de passos tampoc és molt gran i l'\emph{elapsed time} també és moderat. Altres valors positius són el 120, 140 i 170, amb uns valors similars.

Escollim com a valor per defecte del paràmetre $K$ \textbf{110}.


% subsection parametre_k (end)
\newpage
\subsection{Paràmetre $\lambda$} % (fold)
\label{sub:parametre_lambda}

El paràmetre $\lambda$ és el que marca la pendent del refredament de la funció, o sigui com n'és de ràpid aquest refredament un cop comença. A partir dels gràfics observats a les transparències sobre AIMA i el valor per defecte que aquesta classe defineix per $\lambda$, decidim provar els possibles valors entre $0.0005$ i $0.128$ incrementant a base de multiplicar per dos. D'aquesta manera podem contemplar el comportament de la cerca amb un ampli ventall de valors però sense haver de provar-ne milers. Els resultats són bastant clars en el següent gràfic:

\imatge{images/parsa_lambda.png}{Influència del paràmetre $\lambda$}

Degut a la naturalesa exponencial dels valors de l'eix X s'han d'interpretar els resultats exponencialment. S'observa que l'\emph{elapsed time} i el número de passos fins a la solució decrementa molt amb els primers valors de $\lambda$, i que tots els heurístics prenen millors valors amb les $\lambda$ més petites. Per trobar u ncompromís es pot escollir una $\lambda$ de \textbf{0.002} que obté molt bons valors en els heurístics i l'\emph{elapsed time} ha caigut bastant respecte els valors més petits.

\subsection{Resum paràmetres} % (fold)
\label{sub:resum_parametres}

Després de comentar un per un els paràmetres requerits per l'algorisme de Simulated Annealing, mostrem les conclusions d'aquests experiments: Hem partit de la informació bàsica proporcionada pels apunts d'AIMA i el seu propi codi, on hem observat quins valors són raonables aplicar a aquest algorisme. A partir d'aquí hem decidit provar-ne tants com fos oportú per tenir la certesa de trobar l'òptim. Normalment hem provat 20 valors diferents per cada paràmetre, i ens hem quedat amb el que assolia millors resultats tant en el valor dels heurístics com (en menys mesura) els valors de l'\emph{elapsed time} i el número de passos.
\begin{figure}[ht]
  \caption{Taula amb els valors dels paràmetres per Simulated Annealing}
	\label{fig:resumSA}
  \begin{center}
    \begin{tabular}{| l | c | c | c |}
      \hline
      						& \textbf{Valor AIMA}    & \textbf{Valors} & \textbf{Resultat} \\ \hline
      Iteracions màximes 	& 10000 & 500 .. 10000 		& 5500 \\ \hline
      Iteracions per pas 	& 100 	& 20 .. 200 		& 150 \\ \hline
      K 					& 20 	& 10 .. 200 		& 110 \\ \hline
      $\lambda$ 			& 0.045 & 0.0005 .. 0.128 	& 0.02 \\ \hline
    \end{tabular}
  \end{center}
\end{figure}
% subsection resum_parametres (end)

\newpage
% subsection parametre_lambda (end)
% section expSA (end)

\section{Influència de les solucions inicials} % (fold)
\label{sec:expsolini}

	¿Tiene alguna influencia la solución inicial en la calidad de la solució final? Usad solo
	la heurística que minimiza el recorrido total.

Dues configuracions diferents de parades:

\begin{verbatim}
	+-----------------------------+---------+----------+
	|                 Experiment  | Elapsed |    H1    |
	+-----------------------------+---------+----------+
	|       Hill Climbing - Ini 1 |       6 |   230,00 |
	| Simulated Annealing - Ini 1 |    1266 |   226,00 |
	|       Hill Climbing - Ini 2 |       0 |   318,00 |
	| Simulated Annealing - Ini 2 |    1096 |   212,80 |
	+-----------------------------+---------+----------+
\end{verbatim}
	
	Quan s'aplica la solució de qualitat l'algorisme de Hill Climbing ofereix resultats un 36\% més bons. L'algorisme de Simulated Annealing, en canvi, no ofereix millors resultats. Per quèe??????
	
\begin{verbatim}
	+-----------------------------+---------+----------+
	|                 Experiment  | Elapsed |    H1    |
	+-----------------------------+---------+----------+
	|       Hill Climbing - Ini 1 |      15 |   282,00 |
	| Simulated Annealing - Ini 1 |    1277 |   249,20 |
	|       Hill Climbing - Ini 2 |       4 |   246,00 |
	| Simulated Annealing - Ini 2 |    1137 |   247,20 |
	+-----------------------------+---------+----------+
\end{verbatim}	
	
	-> Provar el "Random Restarting Hill Climbing": Consisteix en quedar-se amb el mínim (no la mitjana) dels resultats obtinguts mitjançant l'aplicació de la solució inicial aleatòria.
	
% section expsolini (end)

\section{Com varia la dificultat del problema en funció de K?} % (fold)
\label{sec:expk}

¿Como varia la dificultad del problema dependiendo del número de rutas que se quieren obtener? Usad solo la heurística que minimiza el recorrido total.

% section expk (end)

% Experiment de ponderacio de KH21
\newpage % Per ara ho faig així perquè quedi tot l'experiment en una pàgina i no se'n vagi la imatge porai
\section{Paràmetre de ponderació dels heurístics} % (fold)
\label{sec:expkh}
%!TEX root = ../document.tex

Com que volem una funció heurística que combini els dos paràmetres de qualitat esmentats (apartat ~\ref{sub:heuristic3}
), cal saber com conbinar-los. 

Per tal de calcular una constant de ponderació $k_{h_{21}}$ hem realitzat un experiment que consisteix en observar el valor dels heurístics 1 (recorregut total de les rutes) i 2 (distància entre rutes homogènia) depenent de com els ponderéssim, i evidentment fent que l'algorisme intenti minimitzar la combinació d'ambdós.

Hem utilitzat els paràmetres que han donat millor solucions fins al moment per dur a terme aquest experiment, o sigui:
\begin{itemize}
	\item La solució inicial de qualitat
	\item L'algorisme de Simulated Annealing amb els paràmetres trobats a l'apartat ~\ref{sec:expSA}
	\item Tots els valors de ponderació entre $0.1$ i $1.0$
	\item 10 possibles taulers i 10 iteracions per cada tauler
\end{itemize}

La fórmula per ponderar els heurístics $h_1$ i $h_2$ és la següent:

\begin{center}
	\large
	\[
		h_3 = (1 - k_{h_{21}}) \times h_1 + k_{h_{21}} \times h_2
	\]
\end{center}

\imatge{images/gkh.png}{Valor dels heurístics 1 i 2 amb diferents ponderacions}

Es pot observar a la figura ~\ref{fig:{images/gkh.png}} que clarament els dos heurístics troben un punt mínim a la zona mitja del gràfic, on $k_{h_{21}} \in \{ 0.5 .. 0.6\}$. Per això s'ha fixat per a la resta de proves que el valor ideal de $k_{h_{21}}$ sigui \textbf{0.5}.

% section expkh (end)


\section{Preguntes de l'enunciat} % (fold)
\label{sec:preguntes}

En els experiments anteriors:
\begin{itemize}
\item Quins algorismes responen millor HC o SA?
\item Quins paràmetres del Simulated Annealing fan obtenir millors resultats?
\end{itemize}

% section preguntes (end)
